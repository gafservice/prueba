\documentclass[12pt,letterpaper]{article}

\RequirePackage{xcolor}
\definecolor{tecAzul}{cmyk}{1,0.91,0.33,0.25} % según manual de imagen 2016
\definecolor{tecRojo}{cmyk}{0,0.9,0.86,0}     % según manual de imagen 2016

\renewcommand{\familydefault}{\sfdefault}

\usepackage[spanish]{babel}

\usepackage{titlesec}
\titleformat*{\section}%
{\normalfont\Large\bfseries\color{tecAzul}}
\titleformat*{\subsection}%
{\normalfont\large\bfseries\color{tecAzul}}


\usepackage[tmargin=2cm,bmargin=2cm,lmargin=2cm,rmargin=2cm]{geometry}
\usepackage{textpos}
\usepackage{tikz}
\usepackage{pgfplots}
\usepackage{pgf}

\newcommand{\EstudianteA}{MANUEL ARIAS ABURTO}
\newcommand{\EstudianteB}{SUSANA ASTORGA RODRIGUIEZ}
\newcommand{\EstudianteC}{JOSE ANGULO DURAN}
\newcommand{\EstudianteD}{GERARDO ARAYA FALLAS}

\pgfplotsset{compat=1.17}

\begin{document}
	
\graphicspath{{./}{./fig/}}

%-------------------------- Title section -------------------------------------%

% kit logo
\begin{textblock}{10}[0,0](-0.5,0)
	\begin{flushleft}
		\large 
		Escuela de Ingeniería Electrónica \\
		Licenciatura en Ingeniería Electrónica \\
		EL5841 Taller de Sistemas Embebidos \\
		II Semestre 2021
	\end{flushleft}
\end{textblock}

%Institute and Chair
\begin{textblock}{10}[0,0](2.9,-0.35)
	\begin{flushright}
		\includegraphics[scale=0.8]{Firma_TEC-4.pdf}
	\end{flushright}
\end{textblock}

%% Title %%
\begin{center}
	\vspace{3.5cm}
	{\Large\color{tecRojo} Reporte de Proyecto }
	\par\vspace{1.0cm}
	{\LARGE\bf\color{tecAzul}{Proyecto 1: Diseño de sistemas usando High-Level
Synthesis
}}
	\par\vspace{1.0cm}
	{\large{\EstudianteA, \EstudianteB, \EstudianteC, \EstudianteD} 
	\vspace{0.75cm}}
\end{center}

%------------------------------------------------------------------------------%


\begin{abstract}
Resumen sobre el contenido del reporte y el trabajo realizado como parte del 
proyecto.
\end{abstract}


\section{Computador Host}
Ejemplo de una referencia \cite{legup}.


\section{Características del Toolkit Utilizado OpenCV}
\section{Descripción del flujo de trabajo del toolkit utilizado. OpenCV}
\section{Selección de la aplicación a mostrar}
\section{Lista de dependencias a considerar}
\section{Mapeo de dependencias}
\section{Selección adecuada del Target Machine}
\section{Proceso de síntesis de la imagen}
\section{Proceso de instalación de la imagen sintetizada sobre el sistema de máquinas virtuales}


\section{Conclusiones}
Resumir acá los hallazgos y resultados producto de la experimentación.


% References
\begin{thebibliography}{99}

\bibitem{legup}
Andrew Canis, Jongsok Choi, Mark Aldham, Victor Zhang, Ahmed Kammoona, Jason H. 
Anderson, Stephen Brown, and Tomasz Czajkowski. 2011. \textit{LegUp: high-level 
synthesis for FPGA-based processor/accelerator systems}. In Proceedings of the 
19th ACM/SIGDA international symposium on Field programmable gate arrays (FPGA 
'11). ACM, New York, NY, USA, 33-36. 

\end{thebibliography}


\end{document}
